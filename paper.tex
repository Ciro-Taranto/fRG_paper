 %\documentclass[aps,prb,twocolumn,showpacs,groupedaddress]{revtex4}
\documentclass[aps,prb,twocolumn,showpacs,groupedaddress]{revtex4-1}

%\usepackage[printfigures]{figcaps}
\usepackage[T1]{fontenc}
\usepackage[english]{babel}
\usepackage{epsfig}
\usepackage{graphicx}
\usepackage{color}
\usepackage{amsmath}
\usepackage{bbold}
\usepackage{float}
%\usepackage{caption}
\usepackage{subfigure}
%\usepackage{tikz}

%\newcommand{\bs}[1] {\boldsymbol{#1}}
\newcommand{\bs}[1] {\mathbf{#1}}
\newcommand{\ve}{V}
\newcommand{\fl} [2]{#1^{\Lambda}_{\mathrm{#2} }  }

%\setlength{\parskip}{\plus4mm minus3mm}
%\makeatletter% Set distance from top of page to first float
%\setlength{\@fptop}{-10pt}
%\makeatother

\begin{document}

%\title{Frequency dependent functional renormalization group flow}
\title{Non-separable frequency dependence of two-particle vertex \\ 
        in interacting fermion systems}

\author {D.~Vilardi}
\author{C.~Taranto}
\author{W.~Metzner}
\affiliation{Max Planck Institute for Solid State Research, Heisenbergstrasse 1, D-70569 Stuttgar, Germany}

\date{\today}

\begin{abstract}
We derive functional flow equations for the two-particle vertex and the self-energy in interacting fermion systems which capture the full frequency dependence of both quantities. The equations are applied to the hole-doped two-dimensional Hubbard model as a prototype system with entangled magnetic, charge and pairing fluctuations. Each fluctuation channel acquires substantial dependences on all three Matsubara frequencies, such that the frequency dependence of the vertex cannot be accurately represented by a channel sum with only one frequency variable in each term. At the temperatures we are able to access, the leading instabilities are mostly antiferromagnetic, with an incommensurate wave vector. However, at large doping, a divergence in the charge channel occurs at a finite frequency transfer, if the vertex flow is computed without self-energy feedback. This enigmatic instability was already observed in a calculation by Husemann \emph{et al.} [Phis.~Rev.~B, \textbf{85}, 075121 (2012)], who used an approximate separable ansatz for the frequency dependence of the vertex. We identify a simple mechanism for this instability in terms of a random phase approximation for the charge channel with a frequency dependent effective magnetic interaction as input.
The frequency dependent self-energy is generically affected only mildly by the strongly momentum and frequency dependent two-particle vertex. At the moderate interaction strength where our approach is applicable, we obtain a moderate reduction of the quasi-particle weight and a sizable decay rate with a pronounced momentum dependence. Nevertheless, the self-energy feedback into the vertex flow turns out to be crucial, as it suppresses the unphysical finite frequency charge instability.
\end{abstract}

\pacs{}
\maketitle

\section{Introduction}
\label{sec:introduction}
Exact flow equations describing the evolution of correlation functions upon a successive scale-by-scale evaluation of functional integrals have become a powerful source of new approximation methods in statistical field theory \cite{Berges2002} and in the theory of quantum many-body systems -- especially interacting Fermi systems. \cite{Metzner2012} Among the various versions of these Wilsonian flows, which go under the name {\em functional renormalization group} (fRG), Wetterich's \cite{Wetterich1993} flow equation for the generating functional of one-particle irreducible vertex functions turned out to be particularly efficient. 
While (approximate) non-perturbative solutions of the flow equations are possible for interacting bosons, for fermions one has to rely on an expansion in the fields, truncating the exact hierarchy of flow equations beyond $m$-particle vertex functions of a certain order. One may, however, expand around a non-perturbative starting point, such as the dynamical mean-field solution. \cite{Taranto2014}

The two-particle vertex is a key quantity in any fermionic fRG flow, as it determines the two-particle correlations, leading instabilities, and also the flow of the self-energy. Unfortunately, in quantum systems the two-particle vertex is a difficult object to deal with, due to its dependence on three momentum and frequency arguments. In weakly interacting Fermi systems one may discard the frequency dependence and the momentum dependence perpendicular to the Fermi surface, as these are irrelevant in power counting. This simplification was the basis for early fRG studies of the two-dimensional Hubbard model, using an approximate static parametrization of the vertex, with a momentum dependence discretized by partitioning the Brillouin zone in patches. \cite{Zanchi1996,Halboth2000,Halboth2000b,Honerkamp2001}
Later alternative treatments of the momentum dependence using expansions with form factors were devised.\cite{Husemann2009,Eberlein2013,Eberlein2016}

While irrelevant in power counting, the frequency dependence of the vertex becomes important upon approaching instabilities toward symmetry breaking in the flow.\cite{Husemann2012} Even for weak bare interactions the two-particle vertex becomes large in that regime and acquires singular frequency dependences, for example those associated with the Goldstone boson.\cite{Eberlein2013}
The frequency dependence plays an increasingly important role at strong coupling, as has been confirmed for impurity models,\cite{Kinza2013,Wentzell2016a} and in the dynamical mean field theory (DMFT).\cite{Georges1996,Rohringer2012}
Hence, a proper treatment of the vertex frequency-dependence is mandatory for methods dealing with the interplay between fluctuations in all the channels at strong coupling, such as the combination of DMFT and fRG (DMF$^2$RG), \cite{Taranto2014} and other non-local diagrammatic extensions of the DMFT.\cite{Rohringer2017}

A simplified treatment of the frequency dependence, based on an additive decomposition of the two-particle vertex in pairing, magnetic and charge fluctuation channels, was developed by Husemann et al., \cite{Husemann2012} and applied to an fRG flow for the two-dimensional Hubbard model. They assumed that the dependence of the vertex on the three fermionic frequencies is {\em separable}, that is, each channel depends only on one bosonic transfer frequency, a linear combination of two fermionic frequencies. Already at this level the frequency dependence turned out to be important even at moderate coupling strengths, affecting significantly the energy scale of the leading instabilities. Moreover, for some model parameters an unexpected divergence without any plausible physical interpretation was found in the charge channel at zero momentum and {\em finite} frequency transfer. \cite{Husemann2012}

In this paper we present fRG flows for the two-particle vertex without making any simplifying assumptions or approximations on its frequency dependence. The two-dimensional Hubbard model is used as a prototype fermion system featuring strong and competing fluctuations in several channels. We demonstrate the feasibility, and in some respects, also the necessity of a computation with an unbiased frequency parametrization, even at moderate coupling. Significant {\em non-separable} frequency dependences appear. The various interaction channels do not depend on the bosonic transfer frequencies only, but also on the remaining two fermionic frequencies. We recover the enigmatic charge instability discovered by Husemann et al., \cite{Husemann2012} and reveal its mechanism as the impact of a frequency dependent magnetic interaction on the charge channel.

While a static vertex entails a static self-energy in the one-particle irreducible fRG formalism, the implementation of the full dynamics allows us to compute the frequency (and momentum) dependence of the self-energy. Most interestingly, the feedback of the self-energy into the flow equation for the vertex eliminates the unphysical divergence in the charge channel. This is in contrast with the widespread assumption that the self-energy feedback plays a minor role at moderate interaction strengths.

The paper is structured as follows. In Sec.~\ref{sec:formalism} we will introduce the two-dimensional Hubbard model and the fRG flow equations for the two-particle vertex and the self-energy.
After discussing the channel decomposition and our parametrization of the two-particle vertex  in Sec.~\ref{sec:vertex}, we will move on to the discussion of the main results in Sec.~\ref{sec:results}. Here we identify the leading instabilities, and we discuss the flow of the frequency-dependent vertex. For the charge divergence we provide a transparent diagrammatic explanation, and we finally discuss the momentum and frequency dependence of the self-energy. We draw our conclusions in Sec.~\ref{sec:conclusions}. In the Appendix \ref{sec:FlowEquations} we report all the final expression for the vertex flow equations, while in the Appendix \ref{sec:appPairingChannel} we show the frequency dependence also in the pairing channel.


\section{Formalism}
\label{sec:formalism}

\subsection{Model}

The Hubbard model describes spin-$\frac{1}{2}$ fermions with a density-density interaction:

\begin{equation}
\mathcal{H} = \sum_{i,j,\sigma} t_{ij} c^{\dagger}_{i,\sigma} c_{j,\sigma} + U \sum_{i} n_{i,\uparrow} n_{i,\downarrow}
\end{equation}

where $c^{\dagger}_{i,\sigma}$ and $c_{i,\sigma}$ are, respectively, creation and annihilation operators 
for fermions with spin $\sigma=\uparrow,\downarrow$. We consider the two-dimensional case with square lattice and repulsive interaction $U>0$. The hopping amplitude is restricted to $t_{ij} = t$ for nearest neighbors, $t_{ij}=t'$ for next-to-nearest neighbors and $0$ otherwise.




\subsection{Flow equations}


In the following paragraph we will introduce the functional renormalization group in the implementation that we used, and we will clarify some notational issue about the vertex. 

Generally speaking, the fRG allows to use the renormalization group idea in the functional integral formalism. 
This is done by endowing the non-interacting propagator with an additional dependence on a scale parameter $\Lambda$, which generates an exact functional flow equation with known initial conditions. 

We will apply this approach to the effective action, whose expansions into the fields generates the one-particle irreducible (1PI)  functions. By expanding the functional flow equation one obtains a hierarchy of flow equations for the 1PI functions, involving vertexes of arbitrarily  high orders. 
We will restrict ourselves to the level-two truncation by retaining only the two lowest nonvanishing orders in the expansion, i.e., we consider the flow of the (scale dependent) self-energy $\Sigma^\Lambda$ and of the two-particle 1PI vertex $V^\Lambda$, neglecting the effects of higher order vertexes. 
Hence our approach becomes perturbative, and sums up efficiently, although approximately, the so-called parquet-diagrams. 

, given a cutoff choice with $G_0^{\Lambda_0}=0$We use the energy and momentum conservation to fix one of the arguments of the arguments of the self energy and of the vertex. 
Furthermore we restrict ourselves to the spin-symmetric phase. 
Hence for the self-energy  we only need to consider one function depending on one frequency-momentum argument: 
\begin{equation}
\Sigma^\Lambda_{\sigma\sigma'}(k)=\Sigma(k)\delta_{\sigma,\sigma'}, 
\end{equation}
where $\sigma = \{\uparrow, \downarrow\} $, and $k=(\nu,\mathbf{k})$, $\nu$ being a Matsubara frequency and $\mathbf{k}$ a momentum in the first Brillouin zone. 

For the notation of the two-particle vertex we refer to Fig. (fig), where $k_i=(\nu_i,\mathbf{k_i})$,
and $k_4=(\nu_1+\nu_2-\nu_3,\mathbf{k_1+k_2-k_3})$ can be omitted. 
Furthermore  SU(2)-symmetry guarantees that the vertex does not vanish only for six spin combinations, pairwaise equal under spin inversion:
$
 V^\Lambda_{\uparrow\uparrow\uparrow\uparrow} = V^\Lambda_{\downarrow\downarrow\downarrow\downarrow}$, 
$  V^\Lambda_{\uparrow\downarrow\uparrow\downarrow} = V^\Lambda_{\downarrow\uparrow\downarrow\uparrow}  $, and
$  V^\Lambda_{\uparrow\downarrow\downarrow\uparrow } = V^\Lambda_{\downarrow\uparrow\uparrow\downarrow}$.   
Finally, due to SU(2) symmetry and crossing relation one has: 
\begin{eqnarray}
\nonumber
V^\Lambda_{\uparrow\uparrow\uparrow\uparrow}(k_1,k_2,k_3) &=& V^\Lambda_{\uparrow\downarrow\uparrow\downarrow}(k_1,k_2,k_3)\\&-& V^\Lambda_{\uparrow\downarrow\uparrow\downarrow}(k_1,k_2,k_1+k_2-k_3), \\ 
V^\Lambda_{\uparrow\downarrow\downarrow\uparrow}(k_1,k_2,k_3)& =& -V^\Lambda_{\uparrow\downarrow\uparrow\downarrow}(k_1,k_2,k_1+k_2-k_3).
\end{eqnarray}
This allows us to consider for the vertex only one function of three arguments:  $V^\Lambda_{\uparrow\downarrow\uparrow\downarrow}(k_1,k_2,k_3)=V^\Lambda(k_1,k_2,k_3)$, all the others spin components being obtained by symmetry. 

With these considerations the flow equation for the self energy reads: 
\begin{equation}
\frac{d}{d \Lambda} \Sigma^\Lambda(k)= -\int_q  S^\Lambda(q)\left[2V^\Lambda(k,q,q) -V^\Lambda(k,q,k)\right], 
\end{equation}
with $q=(\omega,\mathbf{q})$ and $k = (\nu,\mathbf{k})$ and we use the notation  $\int_{q} =T\sum_\omega \int_{\mathbf{q}}$, and $\int_{\mathbf{q}}=\int  \frac{d\mathbf{q}}{4\pi^2}$ is the normalized integral over the first Brillouin zone. 
\begin{equation}
 S^\Lambda=\frac{dG^\Lambda}{d\Lambda}\Bigg|_{\Sigma=\mathrm{const}} 
\end{equation}
  is the single-scale propagator; $G^\Lambda=\left[(G_0^\Lambda)^{-1}-\Sigma^\Lambda\right]^{-1}$ is the full propagator,  $G_0^\Lambda$ is the non-interacting Green's function. 
  
 \begin{widetext} 
The vertex flow equation can be written as: 
\begin{align}
 \frac{d}{d\Lambda}V(k1,k2,k3) =  \fl{\mathcal{T}}{pp}(k_1,k_2,k_3) +  
  \fl{\mathcal{T}}{ph}(k_1,k_2,k_3) + \fl{\mathcal{T}}{phc}(k_1,k_2,k_3),
\end{align} 
where:\footnote{The equation for the particle-particle channel is slightly different from the one usually reported in fRG. This is because we took $\fl{V}{} = \fl{V}{\uparrow \downarrow \uparrow\downarrow}$ instead of $\fl{V}{} = \fl{V}{\uparrow\downarrow\downarrow\uparrow}$.  }
\begin{eqnarray} 
\fl{\mathcal{T}}{pp}(k_1,k_2,k_3) &=&-\frac{1}{2} \int_q \fl{{P}}{}(q,k_1+k_2-q) \Big\{  \fl{V}{}(k_1,k_2,k_1+k_2-q)\fl{V}{}(k_1+k_2-q,q,k_3)   \\ 
\nonumber
&&+  \fl{V}{}(k_1,k_2,q)\fl{V}{}(q,k_1+k_2-q,k_3) \Big\} ; \\  
\fl{\mathcal{T} } {ph}(k_1,k_2,k_3) & =& -\int_q \fl{P}{}(q,k_3-k_1+q)
\Big\{ 2 \fl{V}{}( k_1,k_3-k_1+q,k_3)  \fl{V}{}(q,k_2,k_3-k_1+q) \\
\nonumber
&&- \fl{V}{}( k_1,k_3-k_1+q,q)  \fl{V}{}(q,k_2,k_3-k_1+q) - \fl{V}{}( k_1,k_3-k_1+q,k_3)  \fl{V}{}(k_2,q,k_3-k_1+q) \Big\}; \\
\fl{\mathcal{T}}{phc}(k_1,k_2,k_3) & =& \int_q \fl{P}{}(q,k_2-k_3+q) \fl{V}{}(k_1,k_2-k_3+q,q)
\fl{V}{}(q,k_2,k_3).
\end{eqnarray} 
Here we have defined the quantity:
\begin{align}
\fl{ {P }}{}(q,q') &= G^\Lambda(q)S^\Lambda(q') +G^\Lambda(q')S^\Lambda(q).
\end{align} 
For any cutoff choice with $G_0^{\Lambda_0}=0$ the initial condition for the self-and the vertex are, respectively, $\Sigma^{\Lambda_0}=0$ and $V^{\Lambda_0 }= U$. 




\end{widetext} 

\section{Vertex approximation}
\label{sec:vertex}
To parametrize the momentum and frequency dependence of the two-particle vertex, we use the channel decomposition of the vertex introduced by Husemann and Salmhofer, \cite{Husemann2009} where the vertex is written as a sum of the bare interaction and fluctuation induced effective pairing, magnetic and charge interactions.
The function $V^\Lambda(k_1,k_2,k_3)$ is thus decomposed as
\begin{eqnarray}
\nonumber
\ve^{\Lambda}(k_1,k_2,k_3)&=& U - \phi^{\Lambda}_{\mathrm{p}}(k_1+k_2;k_1,k_3)  \\
&+& \phi^{\Lambda}_{\mathrm{m}}(k_3-k_1;k_1,k_2)  \nonumber
 \\ 
 &+&
  \frac{1}{2}  \phi^{\Lambda}_{\mathrm{m}}(k_2- k_3;k_1,k_2) \nonumber \\ & -& \frac{1}{2} \phi^{\Lambda}_{\mathrm{c}}(k_2-k_3;k_1,k_2),
 \label{eq:decomposition}
\end{eqnarray}
with the {\em pairing} channel $\phi_{\mathrm{p}}$, the {\em magnetic} channel $\phi_{\mathrm{m}}$ and the {\em charge} channel $\phi_{\mathrm{c}}$. The first argument of $\phi_{\mathrm{p}}$ is the conserved total momentum and frequency of the particles, while the first argument of $\phi_{\mathrm{m}}$ and $\phi_{\mathrm{c}}$ is a momentum and frequency transfer.
Substituting Eq.~(\ref{eq:decomposition}) into Eq.~(\ref{eq:vertflow}) we obtain: 
\begin{eqnarray}
\nonumber
&&-\dot \phi^{\Lambda}_{\mathrm{p}}(k_1+k_2;k_1,k_3)+ \dot \phi^{\Lambda}_{\mathrm{m}}(k_3-k_1;k_1,k_2)\\&& 
 + \frac{1}{2}  \dot\phi^{\Lambda}_{\mathrm{m}}(k_2- k_3;k_1,k_2)- \frac{1}{2} \dot\phi^{\Lambda}_{\mathrm{c}}(k_2-k_3;k_1,k_2)\\&&=
   \fl{\mathcal{T}}{pp}(k_1,k_2,k_3) +  
  \fl{\mathcal{T}}{ph}(k_1,k_2,k_3) + 
  \fl{\mathcal{T}}{phc}(k_1,k_2,k_3).
  \nonumber
\label{eq:decot}
\end{eqnarray} 
%\end{widetext}
We associate the total momentum argument of $\mathcal{P}^\Lambda_{\rm pp}$ and the momentum transfer argument of $\mathcal{P}^\Lambda_{\rm ph}$ in Eqs.~(\ref{eq:tpp}-\ref{eq:tphc}) to the corresponding arguments of the $\phi_{\mathrm{x}}$ on the right hand side of Eq.~\ref{eq:decomposition}.
This way, it is easy to attribute $\fl{\mathcal{T}}{pp}$ to the flow equation of the only function in Eq. (\ref{eq:decot}) that depends explicitly on $k_1+k_2$: $-\dot\phi_{\mathrm{p}}^\Lambda=\fl{\mathcal{T}}{pp}$. 
The same is true for the particle-hole crossed channel: $\fl{\mathcal{T}}{phc}=\dot\phi_{\mathrm{m}}^\Lambda$.  We associate to the particle-hole diagram the third and fourth term on the left hand side of Eq.~(\ref{eq:decot}): $\fl{\mathcal{T}}{ph}(k_1,k_2,k_3)=\frac{1}{2}  \dot\phi^{\Lambda}_{\mathrm{m}}(k_2- k_3;k_1,k_2) - \frac{1}{2} \dot\phi^{\Lambda}_{\mathrm{c}}(k_2-k_3;k_1,k_2) $.  
%The flow equation of the full vertex is then distributed to the $\phi$-channels. 
%$\phi$ channels are defined by their respictive flow equations:
The flow equations for $\phi_{\mathrm{x}}$ then read: \cite{Husemann2009}
\begin{eqnarray}
\label{eq:phi_p}
 \dot{\phi}_{\mathrm{p}}^{\Lambda}(Q;k_1,k_3) &=&
 -\mathcal{T}^{\Lambda}_{\mathrm{pp}}(k_1,Q-k_1,k_3) , \\
\label{eq:phi_c}
 \dot{\phi}_{\mathrm{c}}^{\Lambda}(Q;k_1,k_2) &=&
 \mathcal{T}^{\Lambda}_{\mathrm{phc}}(k_1,k_2,Q+k_1) \nonumber \\
 && -2\mathcal{T}^{\Lambda}_{\mathrm{ph}}(k_1,k_2,k_2-Q), \\
\label{eq:phi_m}
\dot{\phi}_{\mathrm{m}}^{\Lambda}(Q;k_1,k_2) &=& \mathcal{T}^{\Lambda}_{\mathrm{phc}}(k_1,k_2,Q+k_1) .
\end{eqnarray}
 
Following Refs.~\onlinecite{Husemann2009,Husemann2012}, we address first the momentum dependence. To parametrize the dependence on the fermionic momenta, we use a decomposition of unity by means of a set of orthonormal form factors
$\{f_{l}(\bs{k})\}$.
%The procedure outlined here is described in detail, for example, in %Ref.~\onlinecite{Lichtenstein2017}.
We can then project each channel on a subset of form factors, whose choice is physically motivated.\cite{Husemann2009}
%If one could keep all the form factors the expansion would be exact.

For the pairing channel we keep only $f_{s}(\bs{k}) = 1$ and $f_d(\bs{k})=\cos{k_x} - \cos{k_y}$:
\begin{align}
 \phi^{\Lambda}_{\mathrm{p}}(Q;k_1,k_3) &=
 \mathcal{S}^\Lambda_{\bs{Q},\Omega}(\nu_1,\nu_3) \nonumber \\ 
 &+ f_d\left(\frac{\bs{Q}}{2}-\bs{k}_1\right) f_d\left(\frac{\bs{Q}}{2}-\bs{k}_3\right) \mathcal{D}^\Lambda_{\bs{Q},\Omega}(\nu_1,\nu_3).
\end{align}
A divergence in the channel $\mathcal{S}$ ($\mathcal{D}$) is associated to the emergence of $s$-wave ($d$-wave) superconductivity.\cite{Metzner2012,Platt2013}

For the charge and magnetic channels we restrict ourselves to $f_{s}(\bs{k})=1$ only:
\begin{align}
  \phi^\Lambda_{\mathrm{c}}(Q;k_1,k_2) &= \mathcal{C}^\Lambda_{\bs{Q},\Omega}(\nu_1,\nu_2), \\
  \phi^\Lambda_{\mathrm{m}}(Q;k_1,k_2) &= \mathcal{M}^\Lambda_{\bs{Q},\Omega}(\nu_1,\nu_2).
\end{align}
A divergence of these functions signals $s$-wave instabilities in the charge and magnetic channels, respectively.

Each channel in Eq.~(\ref{eq:decomposition}) contains a (bosonic) linear combination of momenta and frequencies, and two remaining independent fermionic momentum and frequency variables. 
The choice of the mixed notation is natural since the bosonic momenta and 
frequencies play a special role in the diagrammatics.
Indeed, it is the only dependence generated in second order perturbation theory and the main dependence in finite order perturbation theory.
Although one expects a dominant dependence on the bosonic frequency, in particular in the weak coupling regime, we will see that the dependence on the fermionic frequencies can become strong and not negligible, too.
In Refs.~\onlinecite{Husemann2009,Husemann2012}, with no or a simplified frequency dependence, the channel functions are interpreted as bosonic exchange propagators. Such an interpretation is not possible with full frequency-dependence.

The flow equations for the channels $\mathcal{S}$,  $\mathcal{D}$, $\mathcal{C}$ and $\mathcal{M}$ can be derived from the projection of Eqs.~(\ref{eq:phi_p})-(\ref{eq:phi_m}) onto the form factors:
\begin{eqnarray}
\dot{\mathcal{S}}_{\bs{Q},\Omega}^{\Lambda}(\nu_1,\nu_3)  &=& - \int _{\bs{k}_1, \bs{k}_3 } \mathcal{T}^\Lambda_{\mathrm{pp}}(k_1,Q-k_1,k_3), \\ 
\dot{\mathcal{D}}_{\bs{Q},\Omega}^{\Lambda}(\nu_1,\nu_3)  &=& -
\int _{\bs{k}_1,\bs{k}_3}  f_d\left( {\frac{\bs{Q}}{2} - \bs{k}_1} \right ) f_d\left ({\frac{\bs{Q}}{2} - \bs{k}_3} \right)  \nonumber \\ 
 && \times \, \mathcal{T}^\Lambda_{\mathrm{pp}}(k_1,Q-k_1,k_3) , \\
\nonumber
\dot{\mathcal{C}}_{\bs{Q},\Omega}^{\Lambda}(\nu_1,\nu_2) &=& 
\int _{\bs{k}_1,\bs{k}_2}   \mathcal{T}^\Lambda_{\mathrm{phc}}(k_1,k_2,Q+k_1) \\
 && - \, 2\mathcal{T}_{\mathrm{ph}}(k_1,k_2,k_2-Q) , \\ 
\dot{\mathcal{M}}_{\bs{Q},\Omega}^{\Lambda}(\nu_1,\nu_2) & =& 
\int _{\bs{k}_1,\bs{k}_2}  \mathcal{T}^\Lambda_{\mathrm{phc}}(k_1,k_2,Q+k_1) . 
\end{eqnarray}
The final equations are then obtained by substituting the decomposition (\ref{eq:decomposition}) into the equations above, and using trigonometric identities.

As an example we report here the equations for the magnetic channel, while the expressions for the other channels are presented in the Appendix \ref{sec:FlowEquations}:
\begin{widetext}
\begin{equation}
 \dot{\mathcal{M}}^{\Lambda}_{\bs{Q},\Omega}(\nu_1,\nu_2) = 
 \sum_\nu L^{\mathrm{m},\Lambda}_{\mathbf{Q},\Omega}(\nu_1,\nu) 
 P_{\bs{Q},\Omega}^{\Lambda}(\nu) 
 L^{\mathrm{m},\Lambda}_{\mathbf{Q},\Omega}(\nu,\nu_2-\Omega), 
\label{eq:FlowMag}
\end{equation} 	   
with
\begin{equation}
 P_{\bs{Q},\Omega}^{\Lambda}(\omega) = \int_{\bs{p}}
 G^\Lambda(\bs{p},\omega) S^\Lambda(\bs{Q}+\bs{p},\Omega+\omega) +
 G^\Lambda(\bs{Q}+\bs{p},\Omega+\omega) S^\Lambda(\bs{p},\omega),
\label{eq:Pph} 
\end{equation} 
and
\begin{eqnarray} 
\nonumber
 L^{\mathrm{m},\Lambda}_{\bs{Q},\Omega}(\nu_1,\nu_2)
 &=& U + \mathcal{M}^\Lambda_{\bs{Q},\Omega}(\nu_1,\nu_2) 
 + \int_{\bs{p}} \Big \{- \mathcal{S}^\Lambda_{\bs{p},\nu_1+\nu_2}(\nu_1,\nu_1+\Omega)  
 -\frac{1}{2} \mathcal{D}^\Lambda_{\bs{p},\nu_1+\nu_2}(\nu_1,\nu_1+\Omega)
 [\cos(Q_x)+\cos(Q_y)] \\
 &+&\frac{1}{2} \Big[  \mathcal{M}^\Lambda_{\bs{p},\nu_2-\nu_1-\Omega}( \nu_1,\nu_2) 
 - \mathcal{C}^\Lambda_{\bs{p},\nu_2-\nu_1-\Omega}(\nu_1,\nu_2) \Big] 
 \Big \}
\label{eq:Lxph} 
\end{eqnarray}
\end{widetext}
Note that after the momentum integrals in $P$ and $L$ are performed, the right hand side of Eq.~(\ref{eq:FlowMag}) can be expressed as a matrix multiplication in frequency space, where $\Omega$ and $\mathbf{Q}$ appear as parameters.
%\end{widetext}

After this decomposition, the evaluation of the vertex-flow equation, depending on six arguments, is reduced to the flow of the four functions $\mathcal{S}$, $\mathcal{D}$, $\mathcal{C}$, $\mathcal{M}$, each of them depending on three frequencies and one momentum only. In order to compute these equations numerically we discretize the momentum dependence on patches covering the Brillouin zone and truncate the frequency dependence to some maximal frequency value.


\section{Results}
\label{sec:results}
\subsection{Frequency dependence of Vertex}

\begin{itemize}

\item Forward scattering problem seen by Salmhofer

\item Show phase diagram, $\Lambda_{cri}$ vs $x=1-n$, with and without $\Sigma$ 
          (for differenct $t'$)
          
\item Self energy "solve" the problem of charge instability.

\item Suggestion: The charge problem exists also at van Hove filling where, according to the literature, 
           the $\Sigma$ has no effect when Karrasch approximation is taken into account.

\item  Colorplots: Mag and Charge channel

\end{itemize}

While much of the weak coupling momentum structure of the vertex (for the fermionic Hubbard model) is know by means of fRG, its frequency structure has been investigated much less. 
In recent years several results have been obtained for the single impurity Anderson model vertex, both on its own and as essential ingredient for diagrammatic extensions of DMFT. 
Citare: Rohringer, Kinza, Hafermann, Karrasch, Wentzell (and references therein) for the SIAM. Extensions of DMFT: DGA, DF, DMF2RG, Trilex, Quadrilex. 
However a systematic study keeping into account the full frequency dependence and a physically motivated approximation for the momentum dependence, and including fluctuations in all channels is still lacking.

In this persepctive we will present, in the next section, our results obtained by means of fully frequency dependent fRG.
From the methodologic point of view, these results have to be considered as a proof of principle of the feasibility, and in some resepcts of the necessity, of a complete treatment of the frequency dependence of the vertex, with an impact on methods that aim at the study of strong coupling.
From a more physical persepctive we will confirm some results already foreseen by \onlinecite{Husemann2012} with a simpler frequency parametrization. However the study of the frequncy dependence of the verttex will allow us to gain a deeper understanding in these results, in particular the appearance of a \textit{scattering instability}, and a sensitive reduction of the $d$-wave channel. 

Furthermore, a frequency dependent vertex also allows us to compute the frequency dependent self energy, a task that, within fRG, requires heavier approximations whenver one restricts himself to a static vertex.  
We will show that the self-energy feedback in the flow equations is essential to guarantee the consistency between vertex and propagators in the flow equations.
In fact, it turns out that even a Fermi-liquid self-energy can qualitatively change the physical results. 

\paragraph*{Numerical implementation}
We have implemented numerically the flow equations reported in the appendix. 

Due to the different nature of the momentum arguments of self-energy and vertex we have defined two different patching of the irreducible Brillouin zone. 
%The $\phi$-functions depend on a momentum transfer.  
%For the filling and nearest neighbors hopping that we want to describe, the momentum vectors of the most relevant process are $\mathbf{Q}=(0,0)$, important for superconductivity (in the pairing channel) and ferromagnetism (in the magnetic channel); $\mathbf{Q}=(\pi,\pi)$ and its vicinity, relevant for antiferromagnetism and incommensurate antiferromagnetism; the momentum transfer $2k_F$, as defined in Ref.~\onlinecite{Holder2014}, associated to the onset of charge- and spin-density wave instabilities. 
Similarly to what is done in Ref.~\onlinecite{Husemann2009}, the vertex patching describes more accurately the corners around $(0,0)$ and $(\pi,\pi)$, where, for the cases that we will consider, the instability vectors are located.

The situation is completely different for the self-energy, for which the most relevant physics happens in the vicinity of the Fermi surface, at least in the weak coupling regime. Therefore we chose to concentrate the patches along the Fermi surface and in its immediate vicinity, with some further care close to the antinodal points near $(\pi,0)$, relevant for the physics of antiferromagnetism and pseudogap. The representative points are visualized in Fig. (occupation). 

%It is not necessary  that the patching-schemes for vertex and self-energy are connected, but the number of patches used should be roughly of the same order.
In the calculations presented in the following we have used $29$ patches for the vertex and $44$ for the self-energy.

For the practical implementation of the frequency dependence we found convenient to rewrite $\mathcal{S}$, $\mathcal{D}$, $\mathcal{C}$ and $\mathcal{M}$ as function of three bosonic frequencies. 
For each frequency argument we restricted ourselves to at least $40$ positive and $40$ negative Matsubara frequencies. 
We stress that the number of Matsubara frequencies that can be taken into account in the calculation sets the lowest reachable temperature.
  


\subsection{Forward scattering problem}

\begin{itemize}

\item Introduce perpendicular ladder (PL) for charge.

\item Colorplot of charge in PL.

\item Discuss the role of the Bubble at $\boldsymbol{Q}=(0,0)$ and plot it as a function of $\nu$.

\end{itemize}

\subsection{Self energy effects}

\begin{itemize}

\item With self energy feedback, we didn't find any charge instability problem for any parameters range studied.

\item Plot of the Fermi surface based patch scheme.

\item Plot of $\Sigma(i\omega)$ at $\boldsymbol{k}=(\pi,0)$, $\boldsymbol{k}=\boldsymbol{k}_{HS}$ and $\boldsymbol{k}=(\pi/2,\pi/2)$ in frequency space.

\item Plot of $Z_{\boldsymbol{k}}$

\item Plot of occupation with and without $\Sigma$

\end{itemize}

\section{Conclusions}
\label{sec:conclusions}
\begin{itemize}
\item order of the paragraph? shall we discuss first self energy or vertex? 
\item do we want to show more vertex structure in the appendix? 
\end{itemize}

\vspace*{5mm}
\begin{acknowledgments} 
We are grateful to M.~Salmhofer, A.~Eberlein, S.~Andergassen, A.~Toschi for useful discussion. 
We thank O.~Gunnarsson for a critical reading of the manuscript and D.~T.~Mantadakis for comments and suggestions. 
\end{acknowledgments}

\begin{appendix}
\subsection{Flow equations}

Here we derive the flow equations:

\end{appendix}

\bibliography{refs}
%\bibliographystyle{unsrt}
\bibliographystyle{apsrev4-1}


\end{document}
