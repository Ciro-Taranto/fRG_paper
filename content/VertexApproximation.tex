%
% Latex file for Vertex approximation
%
%

%For the parametrization of the vertex, we use bosonic exchange frequencies $\OmPP=\nu_1 + \nu_2$,  
%$\OmPH = \nu_2 - \nu_3$ and $\OmPHC = \nu_3 - \nu_1$.
%If not specify otherwise, we use capital letters for bosonic frequency transfer opposite to fermionic ones.

We start by decomposing the vertex as follows:

\begin{widetext}
\begin{align}
\ve^{\Lambda}(k_1,k_2,k_3) = U - \phi^{\Lambda}_{\mathrm{p}}(k_1+k_2,k_1,k_3) + \phi^{\Lambda}_{\mathrm{m}}(k_3-k_1,k_1,k_3)
 + \frac{1}{2}  \phi^{\Lambda}_{\mathrm{m}}(k_2- k_3,k_1,k_2) - \frac{1}{2} \phi^{\Lambda}_{\mathrm{c}}(k_2-k_3,k_1,k_2),
\end{align}
\end{widetext}

where the physical meaning of each $\phi$ channel is defined by its flow equations:

\begin{align}
\dot{\phi}_{\mathrm{p}}^{\Lambda}(q,k_1,k_3) &= -\mathcal{T}^{\Lambda}_{\mathrm{pp}}(k_1,q-k_1,k_3) , \\
\dot{\phi}_{\mathrm{c}}^{\Lambda}(q,k_1,k_2) &= -2\mathcal{T}^{\Lambda}_{\mathrm{ph}}(k_1,k_2,k_2-1) + \mathcal{T}^{\Lambda}_{\mathrm{phc}}(k_1,k_2,q+k_1) , \\
\dot{\phi}_{\mathrm{m}}^{\Lambda}(q,k_1,k_2) &= \mathcal{T}^{\Lambda}_{\mathrm{phc}}(k_1,k_2,q+k_1) .
\end{align}

It should be emphasized that each channel can be associated with a possible instabily of the 
system. $\phi_{\mathrm{p}}$ is associated to a pairing instability, $\phi_{\mathrm{m}}$ to magnetic 
instabilities and $\phi_{\mathrm{c}}$ to charge instabilities.

%We project each channel into a form factor decomposition, starting from the pairing channel:
We address first the momentum dependence. To this end, we introduce a decomposition of the 
unity by means of a set of orthonormal form factors for the two fermionic momenta $\{f_{l}(\bs{k})\}$ obeying 
the completeness relation:

\begin{equation}
 \int_{\bs{k}}  f_{l}(\bs{k}) f_{m}(\bs{k}) = \delta_{l,m}
\end{equation}

We can then project each channel on a subset of form factors whose choice is physically motivated. 
Let us stress that the full form factor expansion is exact but the truncation introduces an approximation.

For the pairing channel we keep only $f_{s}(\bs{k}) = 1$ and $f_d(\bs{k})=\cos{k_x} - \cos{k_y}$:

\begin{equation}
  \phi^{\Lambda}_{\mathrm{p}}(q,k_1,k_3) =
    \mathcal{S}_{\bs{q}}^{\omega,\nu_1,\nu_3} 
    + f_d\left(\frac{\bs{q}}{2}-\bs{k}_1\right) f_d\left(\frac{\bs{q}}{2}-\bs{k}_3\right) \mathcal{D}_{\bs{q}}^{\omega,\nu_1,\nu_3},
\end{equation}
with the shorthand notation $q=(\omega,\bs{Q})$ and $k_i=(\nu_i,\bs{k}_i)$.

The instability in the channel $\mathcal{S}$ is associated to $s$-wave superconductivity, while $\mathcal{D}$ 
is associated to $d$-wave superconductivity.

For the charge and magnetic channels we restrict ourselves to $f_{s}(\bs{k})=1$ only:
\begin{align}
  \phi_{\mathrm{c}}(q,k_1,k_2) &= \mathcal{C}_{\bs{q}}^{\omega,\nu_1,\nu_2}, \\
  \phi_{\mathrm{c}}(q,k_1,k_2) &= \mathcal{M}_{\bs{q}}^{\omega,\nu_1,\nu_2}
\end{align}
corresponding to instabilities in the charge and magnetic channels, respectively. For now on, for notation simplicity we omit 
the $\Lambda$-dependences of channels $\mathcal{S}$, $\mathcal{D}$, $\mathcal{C}$ and $\mathcal{M}$.

Let us stress that for each channel we have used a different frequency notation.
This consists of one frequency corresponding to the frequency transfered in the specific channel 
and two remaining independent frequencies. At finite temperature the frequency transfer, beeing a sum or a difference of 
two fermionic Matsubara frequencies, is a bosonic Matsubara frequency.

\noindent
The choice of the mixed notation is the most natural since the transferred momentum and 
frequency plays a special role in the diagrammatics.
Indeed, on the one hand, it is the only dependence generated in second order perturbation theory and the main dependence in finite 
order perturbation theory. On the other hand, this notation is convenient in the Bethe-Salpeter equations (cite).

In the fRG literature (cite), where the fermionic frequency dependence is neglected the channel functions above are often losely intepreted as 
mediators of bosonic interactions. Such an interpretation is missing in the presence of all frequencies.

Although one expects a leading dependence in the bosonic frequency, 
in particular in the weak coupling regime, we will see that in some case the dependence on fermionic frequencies can become strong and not negligible.

The flow equations for the channels $\mathcal{S}$,  $\mathcal{D}$, $\mathcal{C}$ and $\mathcal{M}$ can be derived from the projection into form factors of eq. (cite):

\begin{equation}
\dot{\mathcal{S}}_{\bs{q}}^{\omega,\nu_1,\nu_3}  = - \int _{\bs{k}_1} \int_{\bs{k}_2} \mathcal{T}_{\mathrm{pp}}(k_1,q-k_1,k_3)
\end{equation}

\begin{equation}
\dot{\mathcal{D}}_{\bs{q}}^{\omega,\nu_1,\nu_3}  = -
\int _{\bs{k}_1} \int_{\bs{k}_2} f_{\frac{\bs{q}}{2} - \bs{k}_1} \mathcal{T}_{\mathrm{pp}}(k_1,q-k_1,k_3) f_{\frac{\bs{q}}{2} - \bs{k}_3}
\end{equation}

\begin{equation}
\dot{\mathcal{C}}_{\bs{q}}^{\omega,\nu_1,\nu_2}  = 
\int _{\bs{k}_1} \int_{\bs{k}_2}  \mathcal{T}_{\mathrm{phc}}(k_1,k_2,q+k_1) -2\mathcal{T}_{\mathrm{ph}}(k_1,k_2,k_2-q) 
\end{equation}

\begin{equation}
\dot{\mathcal{M}}_{\bs{q}}^{\omega,\nu_1,\nu_2}  = 
\int _{\bs{k}_1} \int_{\bs{k}_2}  \mathcal{T}_{\mathrm{phc}}(k_1,k_2,q+k_1)
\end{equation}
