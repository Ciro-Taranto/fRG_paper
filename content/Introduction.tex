\begin{itemize}
\item Much of the weak coupling momentum structure of the vertex (for the fermionic Hubbard model) is know by means of fRG,its frequency structure has been investigated much less. 
\item In recent years several results have been obtained for the single impurity Anderson model vertex, both on its own and as essential ingredient for diagrammatic extensions of DMFT. 
Citare: Rohringer, Kinza, Hafermann, Karrasch, Wentzell (and references therein) for the SIAM. Extensions of DMFT: DGA, DF, DMF2RG, Trilex, Quadrilex. 
\item A systematic study keeping into account the full frequency dependence and a physically motivated approximation for the momentum dependence, and including fluctuations in all channels is still lacking.

\item Our results show the feasibility, and, in some respects, the necessity  of a complete treatment of the frequency dependence of the vertex, whose impact is particularly large in methods that aim at strong coupling.
\item We will confirm some results already foreseen by \onlinecite{Husemann2012}, who has shown a "forward scattering instability" already with a simpler frequency parametrization.
\item With the study of the frequency dependence of the vertex we understand the appearance of a \textit{scattering instability}.
\item The $d$-wave superconductivity is reduced.  

\item The frequency dependent vertex allows to compute a frequency dependent self energy, often neglected in static fRG.   
\item We will show that the self-energy feedback in the flow equations is essential to guarantee the consistency between vertex and propagators in the flow equations.
\item even a Fermi-liquid self-energy can qualitatively change the physical results. 
\end{itemize}