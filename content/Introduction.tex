Exact flow equations describing the evolution of correlation functions upon a successive scale-by-scale evaluation of functional integrals have become a powerful source of new approximation methods in statistical field theory \cite{Berges2002} and in the theory of quantum many-body systems -- especially interacting Fermi systems. \cite{Metzner2012} Among the various versions of these Wilsonian flows, which go under the name {\em functional renormalization group} (fRG), Wetterich's \cite{Wetterich1993} flow equation for the generating functional of one-particle irreducible vertex functions turned out to be particularly efficient. 
While (approximate) non-perturbative solutions of the flow equations are possible for interacting bosons, for fermions one has to rely on an expansion in the fields, truncating the exact hierarchy of flow equations beyond $m$-particle vertex functions of a certain order. One may, however, expand around a non-perturbative starting point, such as the dynamical mean-field solution. \cite{Taranto2014}

The two-particle vertex is a key quantity in any fermionic fRG flow, as it determines the two-particle correlations, leading instabilities, and also the flow of the self-energy. Unfortunately, in quantum systems the two-particle vertex is a difficult object to deal with, due to its dependence on three momentum and frequency arguments. In weakly interacting Fermi systems one may discard the frequency dependence and the momentum dependence perpendicular to the Fermi surface, as these are irrelevant in power counting. This simplification was the basis for early fRG studies of the two-dimensional Hubbard model, using an approximate static parametrization of the vertex, with a momentum dependence discretized by partitioning the Brillouin zone in patches. \cite{Zanchi1996,Halboth2000,Halboth2000b,Honerkamp2001}
Later alternative treatments of the momentum dependence using expansions with form factors were devised.\cite{Husemann2009,Eberlein2013,Eberlein2016}

While irrelevant in power counting, the frequency dependence of the vertex becomes important upon approaching instabilities toward symmetry breaking in the flow.\cite{Husemann2012} Even for weak bare interactions the two-particle vertex becomes large in that regime and acquires singular frequency dependences, for example those associated with the Goldstone boson.\cite{Eberlein2013}
The frequency dependence plays an increasingly important role at strong coupling, as has been confirmed for impurity models,\cite{Kinza2013,Wentzell2016} and in the dynamical mean field theory (DMFT).\cite{Georges1996,Rohringer2012}
Hence, a proper treatment of the vertex frequency-dependence is mandatory for methods dealing with the interplay between fluctuations in all the channels at strong coupling, such as the combination of DMFT and fRG (DMF$^2$RG), \cite{Taranto2014} and other non-local diagrammatic extensions of the DMFT.\cite{Rohringer2017}

A simplified treatment of the frequency dependence, based on an additive decomposition of the two-particle vertex in pairing, magnetic and charge fluctuation channels, was developed by Husemann et al., \cite{Husemann2012} and applied to an fRG flow for the two-dimensional Hubbard model. They assumed that the dependence of the vertex on the three fermionic frequencies is {\em separable}, that is, each channel depends only on one bosonic transfer frequency, a linear combination of two fermionic frequencies. Already at this level the frequency dependence turned out to be important even at moderate coupling strengths, affecting significantly the energy scale of the leading instabilities. Moreover, for some model parameters an unexpected divergence without any plausible physical interpretation was found in the charge channel at zero momentum and {\em finite} frequency transfer. \cite{Husemann2012}

In this paper we present fRG flows for the two-particle vertex without making any simplifying assumptions or approximations on its frequency dependence. The two-dimensional Hubbard model is used as a prototype fermion system featuring strong and competing fluctuations in several channels. We demonstrate the feasibility, and in some respects, also the necessity of a computation with an unbiased frequency parametrization, even at moderate coupling. Significant {\em non-separable} frequency dependences appear. The various interaction channels do not depend on the bosonic transfer frequencies only, but also on the remaining two fermionic frequencies. We recover the enigmatic charge instability discovered by Husemann et al., \cite{Husemann2012} and reveal its mechanism as the impact of a frequency dependent magnetic interaction on the charge channel.

While a static vertex entails a static self-energy in the one-particle irreducible fRG formalism, the implementation of the full dynamics allows us to compute the frequency (and momentum) dependence of the self-energy. Most interestingly, the feedback of the self-energy into the flow equation for the vertex eliminates the unphysical divergence in the charge channel. This is in contrast with the widespread assumption that the self-energy feedback plays a minor role at moderate interaction strengths.

The paper is structured as follows. In Sec.~\ref{sec:formalism} we will introduce the two-dimensional Hubbard model and the fRG flow equations for the two-particle vertex and the self-energy.
After discussing the channel decomposition and our parametrization of the two-particle vertex  in Sec.~\ref{sec:vertex}, we will move on to the discussion of the main results in Sec.~\ref{sec:results}. Here we identify the leading instabilities, and we discuss the flow of the frequency-dependent vertex. For the charge divergence we provide a transparent diagrammatic explanation, and we finally discuss the momentum and frequency dependence of the self-energy. We draw our conclusions in Sec.~\ref{sec:conclusions}. In the Appendix \ref{sec:FlowEquations} we report all the final expression for the vertex flow equations, while in the Appendix \ref{sec:appPairingChannel} we show the frequency dependence also in the pairing channel.
