\subsection{Frequency dependence of Vertex}

\begin{itemize}

\item Forward scattering problem seen by Salmhofer

\item Show phase diagram, $\Lambda_{cri}$ vs $x=1-n$, with and without $\Sigma$ 
          (for differenct $t'$)
          
\item Self energy "solve" the problem of charge instability.

\item Suggestion: The charge problem exists also at van Hove filling where, according to the literature, 
           the $\Sigma$ has no effect when Karrasch approximation is taken into account.

\item  Colorplots: Mag and Charge channel

\end{itemize}

While much of the weak coupling momentum structure of the vertex (for the fermionic Hubbard model) is know by means of fRG, its frequency structure has been investigated much less. 
In recent years several results have been obtained for the single impurity Anderson model vertex, both on its own and as essential ingredient for diagrammatic extensions of DMFT. 
Citare: Rohringer, Kinza, Hafermann, Karrasch, Wentzell (and references therein) for the SIAM. Extensions of DMFT: DGA, DF, DMF2RG, Trilex, Quadrilex. 
However a systematic study keeping into account the full frequency dependence and a physically motivated approximation for the momentum dependence, and including fluctuations in all channels is still lacking.

In this persepctive we will present, in the next section, our results obtained by means of fully frequency dependent fRG.
From the methodologic point of view, these results have to be considered as a proof of principle of the feasibility, and in some resepcts of the necessity, of a complete treatment of the frequency dependence of the vertex, with an impact on methods that aim at the study of strong coupling.
From a more physical persepctive we will confirm some results already foreseen by \onlinecite{Husemann2012} with a simpler frequency parametrization. However the study of the frequncy dependence of the verttex will allow us to gain a deeper understanding in these results, in particular the appearance of a \textit{scattering instability}, and a sensitive reduction of the $d$-wave channel. 

Furthermore, a frequency dependent vertex also allows us to compute the frequency dependent self energy, a task that, within fRG, requires heavier approximations whenver one restricts himself to a static vertex.  
We will show that the self-energy feedback in the flow equations is essential to guarantee the consistency between vertex and propagators in the flow equations.
In fact, it turns out that even a Fermi-liquid self-energy can qualitatively change the physical results. 


CHANGE BACK MATHBB
   
\subsection{Forward scattering problem}

\begin{itemize}

\item Introduce perpendicular ladder (PL) for charge.

\item Colorplot of charge in PL.

\item Discuss the role of the Bubble at $\boldsymbol{Q}=(0,0)$ and plot it as a function of $\nu$.

\end{itemize}

\subsection{Self energy effects}

\begin{itemize}

\item With self energy feedback, we didn't find any charge instability problem for any parameters range studied.

\item Plot of the Fermi surface based patch scheme.

\item Plot of $\Sigma(i\omega)$ at $\boldsymbol{k}=(\pi,0)$, $\boldsymbol{k}=\boldsymbol{k}_{HS}$ and $\boldsymbol{k}=(\pi/2,\pi/2)$ in frequency space.

\item Plot of $Z_{\boldsymbol{k}}$

\item Plot of occupation with and without $\Sigma$

\end{itemize}