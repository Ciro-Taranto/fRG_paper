In this work we have applied the fRG to the two dimensional Hubbard model, using a form factor decomposition for the momentum arguments of the two-particle vertex, but maintaining intact all the frequency dependence. 
To the best of our knowledge, this is the first time that such an approximation level has been achieved in fRG for fermions on a lattice.
Our results show that a careful treatment of the frequency dependence is not only possible, but also very important.

The bigger complexity of the fully frequency dependent implementation is rewarded, however, by the possibility of better accessing and understanding the frequency structures arising in the flow.
Indeed, we confirm that there exist regions of parameter space where the vertex shows a divergence in the charge channel, as already found in Ref.~\onlinecite{Husemann2012}.
 In this case, we are able to identify a simple set of diagrams that give rise to the above-mentioned divergence, that are likely to originate unexpected features in the charge channel also in other theories that keep into account the vertex frequency dependence and the channel competition.\cite{Stepanov2016}
 
The proper treatment of the vertex frequency-dependence allows for a  calculation of the frequency dependent self-energy, which, in turns, can be feed back into the flow.
 By doing so, we observed that the feedback effect plays an important role, also at the qualitative level, since it suppresses the divergence in the charge channel. 

Given the increasingly importance of the frequency dependence as more correlated regimes are approached, with our work we set the stage for future developments in fRG.
At moderate coupling, like the one treated here, the combination of a frequency dependent vertex and of self-energy feedback allows to revisit and better understand already acquired results.
At strong coupling, more radical measures are required. 
This is what is proposed in DMF$^2$RG,\cite{Taranto2014} where the flow starting point is itself correlated and possibly strongly frequency dependent. 
Therefore consistently keeping into account the vertex frequency dependence can be considered the first step in a path leading to correlation regions that have not yet been accessed by fRG.
%\emph{WM:should we keep fermionic?} (fermionic) fRG. 
 