
\subsection{Model}

The Hubbard model\cite{Hubbard1963} describes spin-$\frac{1}{2}$ fermions with a local interaction:
\begin{equation}
 \mathcal{H} = \sum_{i,j,\sigma} t_{ij} c^{\dagger}_{i,\sigma} c_{j,\sigma}
 + U \sum_{i} n_{i,\uparrow} n_{i,\downarrow} ,
\end{equation}
where $c^{\dagger}_{i,\sigma}$ and $c_{i,\sigma}$ are, respectively, creation and annihilation operators for fermions on site $i$ with spin orientation $\sigma$ ($\uparrow$ or $\downarrow$). We consider the two-dimensional case on a square lattice and repulsive interaction $U>0$ at finite temperature $T$. The hopping amplitude is restricted to $t_{ij} = -t$ for nearest neighbors, $t_{ij}=-t'$ for next-to-nearest neighbors. Fourier transforming the hopping matrix yields the bare dispersion relation
\begin{equation}
 \varepsilon_{\mathbf{k}} =
 -2t \left( \cos{k_x} + \cos{k_y} \right) -4 t' \cos{k_x} \cos{k_y} .
\end{equation}


\subsection{Flow equations}

In the following paragraph we will provide some details about the functional renormalization group,\cite{Metzner2012,Platt2013} defining in particular the notation used for the vertex. 

Generally speaking, the fRG implements the renormalization group idea in the functional integral formalism. 
This is done by endowing the bare action with an additional dependence on a scale-parameter $\Lambda$,\cite{Metzner2012,Platt2013} 
\begin{equation}
 \mathcal{S}^\Lambda[\overline\psi,\psi] =
 -(\overline\psi,{G_0^\Lambda}^{-1}\psi)+\mathcal{S}_{\mathrm{int}},  
\end{equation} 
where $\mathcal{S}_{\mathrm{int}}$ is the interaction part, and $(\overline\psi,\psi)$ summarizes the summation over all the quantum numbers of the fermionic fields  $\overline \psi$ and $\psi$. 
The scale dependence, acquired through the non-interacting propagator $G_0^\Lambda$, generates flow equations (with known initial conditions) for generating functionals defined via functional integrals with the action $\mathcal{S}^\Lambda$, such as the generating functional for the connected Green's function and its Legendre transform, the socalled average effective action.\cite{Wetterich1993}
The final result is recovered for some final $\Lambda$-value restoring the original bare propagator, $G_0^{\Lambda_\mathrm{f}} = G_0$, so that the physical action of interest is recoved.  

We will apply this approach to the effective action, whose expansions in the fields generates the one-particle irreducible (1PI) vertex functions. By expanding the functional flow equation,\cite{Wetterich1993} one obtains a hierarchy of flow equations for the 1PI functions, involving vertices of arbitrarily  high orders. 
We will restrict ourselves to the two-particle level truncation by retaining only the two lowest nonvanishing orders in the expansion, that is, we consider the flow of the self-energy $\Sigma^\Lambda$ and of the two-particle vertex $V^\Lambda$, neglecting the effects of higher order vertices. 
This truncation restricts the applicability of the approach to the weak-to-moderate coupling regime.\cite{Salmhofer2001} 
It can be further shown that, at the two-particle level trunctaion, the fRG sums up efficiently, although approximately, the so-called parquet-diagrams.\cite{Kugler2017}
%\cite{Binz2002,Binz2003,Kugler2017}
  
Due to SU(2) symmetry, the self-energy is diagonal in spin-space: 
\begin{equation}
\Sigma^\Lambda_{\sigma\sigma'}(k)=\Sigma(k)\delta_{\sigma,\sigma'}, 
\end{equation}
where $k=(\nu,\mathbf{k})$, $\nu$ is a fermionic Matsubara frequency and $\mathbf{k}$ a momentum in the first Brillouin zone. 
\begin{figure}[t!]
%\centering
\includegraphics[width=0.2\textwidth]{images/VertexBox.png}
\caption{Notation of the two-particle vertex.} 
\label{fig:notvert} 
\end{figure}

For the notation of the two-particle vertex function $V_{\sigma_1\sigma_2\sigma_3\sigma_4}(k_1,k_2,k_3)$ we refer to Fig.~\ref{fig:notvert}, where $k_i=(\mathbf{k_i},\nu_i)$.
The momentum $k_4=k_1+k_2-k_3$ is fixed by momentum conservation.
The SU(2) spin-rotation symmetry guarantees that the vertex does not vanish only for six spin combinations:
$V^\Lambda_{\uparrow\uparrow\uparrow\uparrow} =
 V^\Lambda_{\downarrow\downarrow\downarrow\downarrow}$, 
$V^\Lambda_{\uparrow\downarrow\uparrow\downarrow} =
 V^\Lambda_{\downarrow\uparrow\downarrow\uparrow}$, and
$V^\Lambda_{\uparrow\downarrow\downarrow\uparrow } =
 V^\Lambda_{\downarrow\uparrow\uparrow\downarrow}$.   
Finally, due to the SU(2) symmetry and crossing relation one has \cite{Rohringer2012} 
\begin{eqnarray}
\nonumber
V^\Lambda_{\uparrow\uparrow\uparrow\uparrow}(k_1,k_2,k_3) &=& V^\Lambda_{\uparrow\downarrow\uparrow\downarrow}(k_1,k_2,k_3)\\&-& V^\Lambda_{\uparrow\downarrow\uparrow\downarrow}(k_1,k_2,k_1+k_2-k_3),
\label{eq:spinsym1}
 \\ 
V^\Lambda_{\uparrow\downarrow\downarrow\uparrow}(k_1,k_2,k_3)& =& -V^\Lambda_{\uparrow\downarrow\uparrow\downarrow}(k_1,k_2,k_1+k_2-k_3).
\label{eq:spinsym2}
\end{eqnarray}
This allows us to express the vertex by only one function of three frequency-momentum arguments: $V^\Lambda(k_1,k_2,k_3)\equiv V^\Lambda_{\uparrow\downarrow\uparrow\downarrow}(k_1,k_2,k_3)$, all the others spin components being obtained by Eqs.~(\ref{eq:spinsym1}-\ref{eq:spinsym2}).\cite{Husemann2009}

The flow equation for the self energy can then be written as \cite{Metzner2012} 
\begin{equation}
\frac{d}{d \Lambda} \Sigma^\Lambda(k)= -\int_p  S^\Lambda(p)\left[2V^\Lambda(k,p,p) -V^\Lambda(k,p,k)\right], 
\end{equation}
with $p=(\mathbf{p},\omega)$ and $k = (\mathbf{k},\nu)$.
We use the notation  $\int_{p} = T \sum_\omega \int_{\mathbf{p}}$, where $\sum_\omega$ is the Matsubara frequency sum, and $\int_{\mathbf{p}}=\int  \frac{d\mathbf{p}}{(2\pi)^2}$ is the normalized integration over the first Brillouin zone. 
\begin{equation}
 S^\Lambda = \left. \frac{dG^\Lambda}{d\Lambda}\right|_{\Sigma^{\Lambda}=\mathrm{const}} 
\end{equation}
is the socalled single-scale propagator, and ${G^\Lambda}$ is the full propagator, which is related to the bare propagator and the self-energy by the Dyson equation
$(G^\Lambda)^{-1} = (G_0^\Lambda)^{-1} - \Sigma^\Lambda$. 
  
\begin{widetext} 
The flow equation for the vertex can be written as \cite{Metzner2012, Husemann2009}
\begin{align}
\label{eq:vertflow}
 \frac{d}{d\Lambda}V^\Lambda(k_1,k_2,k_3) = \fl{\mathcal{T}}{pp}(k_1,k_2,k_3) +  
  \fl{\mathcal{T}}{ph}(k_1,k_2,k_3) + \fl{\mathcal{T}}{phc}(k_1,k_2,k_3),
\end{align} 
where \footnote{The equation for the particle-particle channel is slightly different from the one usually reported in fRG, see, e.g., Ref.~\onlinecite{Husemann2009}. This is because we took $\fl{V}{} = \fl{V}{\uparrow \downarrow \uparrow\downarrow}$ instead of $\fl{V}{} = \fl{V}{\uparrow\downarrow\downarrow\uparrow}$.}
\begin{eqnarray}
\label{eq:ppT} 
\fl{\mathcal{T}}{pp}(k_1,k_2,k_3) &=&-\frac{1}{2} \int_p \fl{\mathcal{P}}{}(p,k_1+k_2-p) \Big\{  \fl{V}{}(k_1,k_2,k_1+k_2-p)\fl{V}{}(k_1+k_2-p,p,k_3) 
\label{eq:tpp} 
   \\ 
\nonumber
&&+  \fl{V}{}(k_1,k_2,p)\fl{V}{}(p,k_1+k_2-p,k_3) \Big\} ; \\  
\label{eq:tph} 
\fl{\mathcal{T} } {ph}(k_1,k_2,k_3) & =& -\int_p \fl{\mathcal{P}}{}(p,k_3-k_1+p)
\Big\{ 2 \fl{V}{}( k_1,k_3-k_1+p,k_3)  \fl{V}{}(p,k_2,k_3-k_1+p) \\
\nonumber
&&- \fl{V}{}( k_1,k_3-k_1+p,p)  \fl{V}{}(p,k_2,k_3-k_1+p) - \fl{V}{}( k_1,k_3-k_1+p,k_3)  \fl{V}{}(k_2,p,k_3-k_1+p) \Big\}; \\
\label{eq:tphc}
\fl{\mathcal{T}}{phc}(k_1,k_2,k_3) & =& \int_p \fl{\mathcal{P}}{}(p,k_2-k_3+p) \fl{V}{}(k_1,k_2-k_3+p,p)
\fl{V}{}(p,k_2,k_3).
\end{eqnarray} 
The subscripts $\mathrm{pp}$, $\mathrm{ph}$ and $\mathrm{phc}$ stand respectively for \textit{particle-particle}, \textit{particle-hole} and \textit{particle-hole crossed}.
Here we have defined the quantity
\begin{align}
 \mathcal{P}^\Lambda(p,p') &= G^\Lambda(p)S^\Lambda(p') + S^\Lambda(p) G^\Lambda(p'),
%\fl{ \mathcal{P}}{}(p,Q+p) &= G^\Lambda(p)S^\Lambda(Q+p) +G^\Lambda(p+Q)S^\Lambda(p),
\end{align} 
which is the scale-derivative, at fixed self energy, of the product of two Green's functions. %$Q$ is the frequency and momentum transfer.




