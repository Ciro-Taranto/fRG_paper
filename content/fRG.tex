\subsection{Flow equations}

\begin{itemize} 
\item General consideration about functional renormalizaton group: 1PI, truncation, 
\item vertex: notation, momentum and energy conservation, spin. Spin symmetry 

\end{itemize}

In the following paragraph we will introduce the functional renormalization group in the implementation that we used, and we will clarify some notational issue about the vertex. 

Generally speaking, the fRG allows to use the renormalization group idea to approach functional integrals. 
This is done by endowing the non-interacting propagator with an additional dependence on a scale parameter $\Lambda$, which generates an exact functional flow equation with known initial conditions. 

We will apply this approach to the effective action, whose expansions into the fields generates the one-particle irreducible (1PI)  functions. By expanding the functional flow equation one obtains a hierarchy of flow equations for the 1PI functions, involving vertexes of arbitrarily  high orders. 
We will restrict ourselves to the level-two truncation by retaining only the two lowest nonvanishing orders in the expansion, i.e., we consider the flow of the (scale dependent) self-energy $\Sigma^\Lambda$ and of the two-particle 1PI vertex $V^\Lambda$, neglecting the effects of higher order vertexes. 
Hence our approach becomes perturbative, and sums up efficiently, although approximately, the so-called parquet-diagrams. 

\subparagraph*{Symmetry considerations} We use the energy and momentum conservation to fix one of the arguments of the arguments of the self energy and of the vertex. 
Furthermore we restrict ourselves to the spin-symmetric phase. 
Hence for the self-energy  we only need to consider one function depending on one frequency-momentum argument: 
\begin{equation}
\Sigma^\Lambda_{\sigma\sigma'}(k)=\Sigma(k)\delta_{\sigma,\sigma'}, 
\end{equation}
where $\sigma = \{\uparrow, \downarrow\} $, and $k=(\nu,\mathbf{k})$, $\nu$ being a Matsubara frequency and $\mathbf{k}$ a momentum in the first Brillouin zone. 

For the notation of the two-particle vertex we refer to Fig. (fig), where $k_4=(\nu_1+\nu_2-\nu_3,\mathbf{k_1+k_2-k_3}$ can be omitted. 
Furthermore spin conservation and SU(2) guarantee that the vertex does not vanish only for six spin combinations, pairwaise equal under spin inversion: $V^\Lambda_{\sigma\sigma\sigma\sigma}$, $V^\Lambda_{\sigma\sigma\overline\sigma\overline\sigma}$ and $V^\Lambda_{\sigma\overline\sigma\overline\sigma\sigma}$. \emph{should we specify the meaning of $\overline \sigma$ or is it obvious?}
Finally, using the crossing relation we obtain: 
\begin{eqnarray}
V^\Lambda_{\sigma\sigma\sigma\sigma}(k_1,k_2,k_3) &=& V^\Lambda_{\sigma\sigma\overline\sigma\overline\sigma}(k_1,k_2,k_3)- V^\Lambda_{\sigma\sigma\overline\sigma\overline\sigma}(k_1,k_2,k_4), \\ 
V^\Lambda_{\sigma\overline\sigma\overline\sigma\sigma}(k_1,k_2,k_3)& =& -V^\Lambda_{\sigma\sigma\overline\sigma\overline\sigma}(k_1,k_2,k_1+k_2-k_3). 
\end{eqnarray}
This allows us to consider only one spin component for the vertex $V^\Lambda(k_1,k_2,k_3)=V^\Lambda_{\sigma\sigma\overline\sigma\overline\sigma}(k_1,k_2,k_3)$, all the others being obtained by symmetry. 

With these considerations the flow equation for the self energy reads: 
\begin{equation}
\frac{d \Sigma^\Lambda(k)}{d \Lambda}= -T\sum_{\omega}\int\frac{d\mathbf{q}}{4\pi^2} S^\Lambda(q)\left[2V^\Lambda(k,q,q) -V^\Lambda(k,q,k)\right], 
\end{equation}
with $q=(\omega,\mathbf{q}$ and $k = (\nu,\mathbf{k})$. 
\begin{equation}
S^\Lambda=-(G^\Lambda)^2\frac{d(G_0^\Lambda)^{-1}}{d\Lambda}
\end{equation} is the single-scale propagator, $G_0^\Lambda$ the non-interacting Green's function and $G^\Lambda=\left[(G_0^\Lambda)^{-1}-\Sigma^\Lambda\right]^{-1}$ the full propagator. 

The vertex flow equation can be written as: 
\begin{equation}
 \frac{dV(k1,k2,k3)}{d\Lambda} = \mathcal(T}^\Lambda_{\mathrm{pp}}(k_1,k_2,k_3)
\end{equation} 
\emph{ok, lo so, sono un rompipalle, possiamo ricontrollare il fattore un mezzo nel canale particle particle che non sta in nessun caso in FRG?}.
